\chapter{Simulator limits} \label{limits}

In this chapter we explore theoretical limits of simulators both in terms of speed, and what is even feasible.
\\

Throughout this chapter, the baseline model will consist of 4,773 actors comprised 128 racks connected to 37 backbone switches and 37 servers each through bidirectional 10Gbps links.
The full bandwidth of the network is therefore $2\times128\times(37+37)\times10Gbps = 190Tbps$, over 10 simulated seconds this corresponds to up to almost 2,000Tb of data.
The packet size will be assumed to be 1,500~bytes, the TCP standard. \\
Note that network speeds are typically measured in bits per second, not bytes.
For more details on the structure of datacenters see section \ref{model-dc}.

\section{Memory feasability} \label{limits-mem}

The main limiting factor in the feasability of simulations is memory consumption.
Speed is typically not a concern since it is always possible, if frustrating, to wait longer.
Storing the results of a large simulation could also be limiting but disk space is relatively cheap.

There are three main sources of memory usage in this simulation: events, packets, and flows.
Although there are nearly 5,000 actors, they each (without their events, packets, and flows) take up fairly little memory.
Even if they each took up a full 10 kilobytes, this would correspond to 50MB of RAM, barely noticeable on today's computers.

\paragraph{Packets}
A fully saturated network on the model above will result in 16 billion packets being sent for every simulation second.
Each packet contains information about its source, destination, size, the flow it belongs to, and so on...
%mark

\paragraph{Events}

\paragraph{Flows}


\section{Speed} \label{limits-speed}

\paragraph{Metrics}
In order to more easily compare the speed of different simulators, it is useful to talk about the amount of simulated data moved per real second.
Since this metric has units of data over time, I will refer to it as the simulator's bandwidth.
As a useful reference, a simulation bandwidth of 100Gbps results in a running time of 5.5h for a typical 10s run.

A more standard metrics is the number of events per second that the simulator executes.
It is less useful here since it doesn't directly translate to how long the simulator runs, it can also be highly dependent on the type of events being processed.


\subsection{Processing power} \label{limits-cpu}

In the worst case scenario, the network is fully saturated, meaning that the 190Tbps of the network are filled with packets.
This corresponds to 16 billion packets per simulated second (16Gp/s), meaning at least that many events.

If a packet takes around 100 CPU cycles to process on average, which is an optimistic guess: a single L2 cache access takes around 10 cycles, on a 2GHz CPU that corresponds to 50ns of processing per packet, or 20 million simulated packets per real CPU-second. % TODO cite cache speed
Although this may seem slow compared to the 16Gp/s of the network under simulation, this corresponds to a simulation bandwidth of 240Gbps.

Optimistically, across multiple CPUs, we can hope to multiply this effect.
My laptop's 8 cores may therefore dream of running a simulation at nearly 2Tbps.

Realistically, 100 cycles per packet is optimistic, and in the case of a parallel simulaiton, the synchronization between cores becomes expensive: a typical shared lock operation alone typically taking 20-30ns. %TODO cite
These numbers do allow us to understand an upper bound of what is possible.



\subsection{I/O} \label{limits-io}

The slowest component of a computer is always the IO interface, and the entire point of running simulations is to have a record of what happened during the running of it.
It is therefore necessary to log particular events during the running of the simulation.

Since I am interested in finding upper bounds, I will be particularly generous in the encoding schemes proposed.
In practice, I expect these data to take up more space than I propose, but these calculations serve to limit our dreams and not necessarily indicate what is realistic.

\paragraph{Packet-level logging}
The highest level of detail can be measured by logging every single packet movement.
The bare minimum needed would be to log some form of packet identifier and a source and destination.
Given 5,000~hosts, the source and destination can each have 16~bits of a 32~bit integer.
However, uniquely identifying the packet is tricky, but a 64~bit integer should theoretically be enough.
This results in 12~bytes of data per packet movement, and a logging bandwidth of almost 200GB per simulated second.

The fastest SSDs advertise writing rates around 500MBps, which if fully utilized yields a maximum simulation rate around 500Gbps.
However, the amount of data generated under this scheme is large: it is typical to run dozens of simulations for a single plot, and under this logging scheme and nearly magical encoding, each run would produce nearly 2TB of data.

However, this calculation illustrates that IO is not as much of a limit as might be expected: 500Gbps is twice as high as the single-CPU limit of section \ref{limits-cpu}.
Fortunately for us, IO is even less of an issue if we log less data.

\paragraph{Flow-level logging}
We can significantly reduce our IO needs if instead of logging each packet, we log only the flows, along with its source, destination, size, and duration.
The source and destination can again both share a 32~bit integer.
However, the size of the flow in simulation can vary from a few KB to GBs, a range that can barely fit in 32~bit values.
Time in the simulator is measured down to the nanosecond, but such precision is not required, allowing it to also fit in a 32~bit value.
This brings the overall size of logging a single flow to three 32~bit values, or 12~bytes.

The last value that is required to compute this limit is the average flow size.
However, this is highly dependent on the traffic distribution (see \ref{model-traffic}).
The datamining workload has an average flow size around 62Mb, a small distribution we use internally has an average of 1.5Mb, while another one has an average of 1,100Mb.

With the smallest average flow size of 1Mb, we will be producing over 2GB per simulated second of log data, which, consumed at 500MBps supports simulation speeds up to 40Tbps.
Datamining can be simulated up to 2,500Tbps, and the largest distribution up to 45Pbps.
In other words, IO is not the limiting factor.


