\chapter{Datacenter network model} \label{model}

\section{Topologies} \label{model-topology}

Most datacenters, and models thereof \cite{mellette_rotornet_2017}\cite{handley_re-architecting_2017}\cite{mellette_expanding_2019} (more?) are organized in hundreds or thousands of racks containing servers.
There are anywhere from 5 to 50 servers in a rack, each connected to a switch typically at the top, giving it the common name "Top of Rack switch" or ToR.
A "small" datacenter may have around 5,000 machines, large ones up to 250,000 or more. %cite

The racks themselves are connected together through different fabrics, and switch technologies.
The topology and technologies behind these is extremely varied \cite{kassing_beyond_2017}, and the subject of extensive research (Xpander, rotornet, opera).

The typical bandwidth vary from 10Gbps to 100Gbps on each of those links.
Some networks assume different link speeds between the ToRs than from server to ToR, others are uniform.

The latencies involved are often extremely small, on the order of 10s of nanoseconds.

%TODO talk about reconfigurable (and rotor) switches

\section{Traffic modelling} \label{model-traffic}

There are a few well studied datacenter traffic models (datamining, websearch, Chen Stefan Manya paper, more), with a common takeaway being that datacenter traffic is hard to predict.

Traffic distribution can be broken down into multiple dimensions: overall load, skeweness, and flow size distribution.

Overall load corresponds to how much traffic is being offered onto the network relative to how much capacity the network has.
It is important to note that this does not mean that the network is able to 100\% load, most networks are overwhelmed far before.
The reason for this is simple: if on average, it takes 4 hops to get to the destination, traffic is essentially being sent 4 times, making any load above 25\% impossible to satisfy.

The traffic may be fairly uniform, each sender and receiver being involved in traffic all around the datacenter, or very concentrated, with a few "hot" racks being very active with each other, the rest of the datacenter quiet.
Describing these patterns can become quite complex, as there may be multiple "hot" regions mostly not interacting with each other.
Throoughout this thesis, and in most of the literature, we tend to be concerned about uniform traffic, and heavily skewed traffic. % is this true??

Finally, the flow size distribution corresponds to how big each flow is expected to be \cite{alizadeh_data_2010}.
Some workloads, such as websearch, consists mostly of many small flows.
Others, such as datamining, are comparatively much bigger.
Machine working loads also typically create a large demand, for example in a ring, using all the networking resources available.

\section{Failures} \label{failures} % unsure if necessary

Due to the presence of such a large number of devices, failures become extremely common. %cite
To illustrate this, if a datacenter has 100,000 machines and a component, say a motherboard has an average lifetime of 10 years, we should expect a failure every hour.
Although ubiquitous in the day to day of a datacenter, it is not uncommon to not include them as part of the model, and do a separate analysis for performance under failure. % What am I trying to say here??

\section{Level of simulation} \label{model-level}

Flow vs packet