\chapter{Introduction} \label{intro}

\section{Models and simulators}

% blah, not sure I like this intro
Models and simulations are ubiquitous in society today.
Their most visible use currently is in trying to understand and predict the spread of COVID-19, but they are used in almost every research discipline as a cheap way of testing an idea in software.

To help us understand the world, we build simplified models which we hope are a good representation of it.
A model may then be implemented into a simulation.
These concepts are worth distinguishing: a simulator may perfectly simulate a model, but yield erroneous results due to the model being too imprecise.
This distinction lies at the heart of this thesis: the objective is to make a fast simulator, using datacenter networking models as an example.
The contribution of this work is almost entirely in the simulator, not the model.

Discrete-event simulations in particular typically consist of actors reacting to events.
Each event may then generate, or not, new events, affecting some actors.
A classical example could be a car-wash, events being consumers arriving, potentially queuing, getting sent to a cleaning station, and leaving once done.
Depending on the arrival pattern, queues may build up in front of the station, a more complex model might even take into account customers giving up and leaving the queue. %cite old paper
The actors would be the queue and the cleaning stations. Events correspond to cars arriving and departing from each of those points.
A data network is similar, cars becoming packets or flows, stations becoming servers and switches.

Discrete events simulators stand in contrast to continuous simulators, which typically implement models better described by differential equations.
Typical models would be fluid dynamic, solar systems, protein folding, and the like. % TODO find refs