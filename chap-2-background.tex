\chapter{Background and related work} \label{background}

In this chapter I describe...

\section{Centralized discrete event simulators} \label{centralized-sim}

Event driven simulators typically consist of multiple actors and a scheduler, sometimes referred to as the event queue. %todo ref(s)
The scheduler is the driver of the simulation, its operation fairly straightforward: execute the next event, insert any new events at an appropriate place in the queue, launch the next event.
The relatively simple nature of the scheduler makes it so that it can usually be implemented in relatively few lines of code.
For the sequential simulator, RotorSim\cite{brode-roger_nibriviarotorsim_2020} described in \S \ref{rotorsim}, the loop and event processing code takes up around 30 lines of python.

% not happy with this paragraph structure
Large simulation models can generate millions or billions of events through their lifetime, causing the queue to continuously hold thousands or more events at a time, see the results section \S\ref{results} for more details.
In addition, since events are typically fast to process, the queue is also continuously being accessed and modified and can easily become a bottleneck.
The large size and frequent access of the queue, force careful consideration of the data structures used to implement the queue/

The queue has to support two operations: retrieving the smallest element currently in the queue, and inserting new events.
This naturally leads to using a min-heap, allowing for $O\left(\log n\right)$ insertion and removal of the smallest element.
Although there exists slightly faster "monotonic heaps", such as the Ladder Queue \cite{tang_ladder_2005}, that are able to further either the complexity or the constant factor, they are usually complex to implement and tune \cite{furfaro_adaptive_2018}, and don't have available efficient implementations.
Standard heaps however are common enough that most every language has a well-optimized implementation, often making them easier to use.

Fundamentally, these simulations are centralized: the scheduler drives the model forward.
Some parallelization is possible if many events are scheduled at the same virtual time *and* the actors are able to deal with that.
Even if the simulation has many events executing in parallel, we run into issues when they try to schedule new events: the scheduler needs to manage multiple new events being enqueued concurrently.
In my experience, the overhead of managing these concurrent operations results in worse performance overall.

\paragraph{Preservation of causality}

In order for a simulation to be correct, it is essential that before any event is processed, all past events it depends on have taken effect.
The dependency on these prior events may be obvious: a packet needs to arrive at a router before it can be sent from said router, or not: in order to know where in the queue to place the arriving packet, it is necessary to know \emph{every} other arriving packet, even if it is from a different source and to a different destination, a previous packet may fill up the router's available memory.

The centralized simulator ensures causality by processing events in absolute time order: by the time any event is executed, all events in its past have fully completed.
Events happening simultaneously are typically assumed to not affect each other and can therefore happen in any order.

\paragraph{Guarantee of progress}
This simulator will always make progress as long as there is another event to process, trivially allowing it to keep making progress until there is none left.



\section{Parallel discrete event simulators} \label{pdes}

The main difficulty in making event simulators parallel lies in guaranteeing causality and progress.
It is often easy to guarantee one without the other.

In order to better understand the opportunity, or lack thereof, for parallelization, it is necessary to better understand that causality requirements imposed by the model.
A causality graph consists of a node for each event in the simulation, and a directed edge between from a dependency to the future event.
Any topological sort of the resulting graph is a valid execution order of the events.
Since the direction of the graph is always to the future, sorting the events by time is trivially correct.
This is the approach taken by the centralized simulator.

If a pair of events are direct ancestors/descendants of each other, i.e. there is a direct line going from one to the other, then there is a dependence between each other and they may not be executed simultaneously.
Conversely, if there is no direct connection between two events, they may be executed simultaneously.
This does not prevent events from sharing a common ancestor or descendant.

These events that are neither ancestor nor descendants are essentially independent of the current event, they can happen before, or after, or simultaneously, it will not affect the correctness of the simulation.
This is very similar to events outside the light-cone in physics: the happening, or not, of events outside the light cone have no causal relationship with the current point.

The correctness challenge is therefore reduced to knowing whether the next event to process is safe, i.e. all of its ancestors have been processed.

\subsection{Causality preservation} \label{causality}

An old algorithm, described as early as 1986\cite{misra_distributed_1986}, TODO...

\subsection{Conservative scheduling and null-message passing} \label{null-messages}

In order to TODO

\subsection{Other approaches}
\paragraph{Optimistic scheduling} \label{optimistic-scheduling}

It is also possible to allow actors to process events as fast as they can, possibly resulting in causality violations.
In order to maintain correctness, the simulation needs to have a cancellation mechanism, some use anti-events \cite{} or checkpointing \cite{}. %todo cite
This hopeful approach is the source of its name.

The cost of optimistic scheduling comes from the overhead necessary to allow for rollbacks, and the execution of these rollbacks.
\emph{TimeWarp} \cite{} is a popular example of an optimistic scheduling algorithm.

Trying and rolling back is a common technique in computer engineering.
Databases use optimistic concurrency control instead of locks to rollback transactions in case of a conflict \cite{dragojevic_no_2015}.
CPUs are continuously predicting the next instruction to allow for higher speeds, but also need a cancelling mechanism if the prediction turns out incorrect.

\paragraph{Loss of accuracy}
Another strategy could be to let execute events in the wrong order, possible yielding incorrect results.
Results from this strategy are harder to justify: incorrect results may come from an unlucky run or simulator.
In this thesis, correctness is an important design goal: it makes it easier to claim the results of research are good, and not just a quirk of the simulator.
There may be cases where loss of accuracy is acceptable, or can be bounded sufficiently to make it useful.
